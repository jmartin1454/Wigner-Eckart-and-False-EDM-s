%% This is file `elsarticle-template-1-num.tex',
%%
%% Copyright 2009 Elsevier Ltd
%%
%% This file is part of the 'Elsarticle Bundle'.
%% ---------------------------------------------
%%
%% It may be distributed under the conditions of the LaTeX Project Public
%% License, either version 1.2 of this license or (at your option) any
%% later version.  The latest version of this license is in
%%    http://www.latex-project.org/lppl.txt
%% and version 1.2 or later is part of all distributions of LaTeX
%% version 1999/12/01 or later.
%%
%% Template article for Elsevier's document class `elsarticle'
%% with numbered style bibliographic references
%%
%% $Id: elsarticle-template-1-num.tex 149 2009-10-08 05:01:15Z rishi $
%% $URL: http://lenova.river-valley.com/svn/elsbst/trunk/elsarticle-template-1-num.tex $
%%
\documentclass[preprint,12pt]{elsarticle}

%% Use the option review to obtain double line spacing
%% \documentclass[preprint,review,12pt]{elsarticle}

%% Use the options 1p,twocolumn; 3p; 3p,twocolumn; 5p; or 5p,twocolumn
%% for a journal layout:
%% \documentclass[final,1p,times]{elsarticle}
%% \documentclass[final,1p,times,twocolumn]{elsarticle}
%% \documentclass[final,3p,times]{elsarticle}
%% \documentclass[final,3p,times,twocolumn]{elsarticle}
%% \documentclass[final,5p,times]{elsarticle}
%% \documentclass[final,5p,times,twocolumn]{elsarticle}

%% The graphicx package provides the includegraphics command.
\usepackage{graphicx}
%% The amssymb package provides various useful mathematical symbols
\usepackage{amssymb}
%% The amsthm package provides extended theorem environments
\usepackage{amsthm}
\usepackage{amsmath}

%% The lineno packages adds line numbers. Start line numbering with
%% \begin{linenumbers}, end it with \end{linenumbers}. Or switch it on
%% for the whole article with \linenumbers after \end{frontmatter}.
\usepackage{lineno}

\usepackage{siunitx}

%% natbib.sty is loaded by default. However, natbib options can be
%% provided with \biboptions{...} command. Following options are
%% valid:

%%   round  -  round parentheses are used (default)
%%   square -  square brackets are used   [option]
%%   curly  -  curly braces are used      {option}
%%   angle  -  angle brackets are used    <option>
%%   semicolon  -  multiple citations separated by semi-colon
%%   colon  - same as semicolon, an earlier confusion
%%   comma  -  separated by comma
%%   numbers-  selects numerical citations
%%   super  -  numerical citations as superscripts
%%   sort   -  sorts multiple citations according to order in ref. list
%%   sort&compress   -  like sort, but also compresses numerical citations
%%   compress - compresses without sorting
%%
%% \biboptions{comma,round}

% \biboptions{}

\journal{Journal Name}

\begin{document}

\begin{frontmatter}

%% Title, authors and addresses

\title{Wigner-Eckart Theorem and the False EDM of Hg}

%% use the tnoteref command within \title for footnotes;
%% use the tnotetext command for the associated footnote;
%% use the fnref command within \author or \address for footnotes;
%% use the fntext command for the associated footnote;
%% use the corref command within \author for corresponding author footnotes;
%% use the cortext command for the associated footnote;
%% use the ead command for the email address,
%% and the form \ead[url] for the home page:
%%
%% \title{Title\tnoteref{label1}}
%% \tnotetext[label1]{}
%% \author{Name\corref{cor1}\fnref{label2}}
%% \ead{email address}
%% \ead[url]{home page}
%% \fntext[label2]{}
%% \cortext[cor1]{}
%% \address{Address\fnref{label3}}
%% \fntext[label3]{}


%% use optional labels to link authors explicitly to addresses:
%% \author[label1,label2]{<author name>}
%% \address[label1]{<address>}
%% \address[label2]{<address>}

\author[um]{W.~Klassen}
\author[uw,um]{J.W.~Martin}
\author[lpc]{G.~Pignol}

\address[um]{Physics and Astronomy, University of Manitoba, Winnipeg, MB R3T 2N2}
\address[uw]{Department of Physics, The University of Winnipeg, Winnipeg, MB R3B 2E9}
\address[lpc]{LPC Grenoble}

\begin{abstract}

\end{abstract}

\begin{keyword}
false electric dipole moment \sep trapped particles \sep neutron electric dipole moment
%% keywords here, in the form: keyword \sep keyword

%% MSC codes here, in the form: \MSC code \sep code
%% or \MSC[2008] code \sep code (2000 is the default)

\end{keyword}

\end{frontmatter}

%%
%% Start line numbering here if you want
%%
%\linenumbers

%% main text
\section{False electric dipole moments for particles in traps}
\label{sec:intro}

In the most precise neutron electric dipole moment experiments,
ultracold neutrons are placed in a bottle containing either parallel
or antiparallel electric $E$ and magnetic $B_0$ fields.  Their spin
precession frequency $\nu_\pm$ is measured
\begin{equation}
h\nu_\pm=\gamma_nB_0\pm 2d_nE
\end{equation}
for each of the parallel ($+$) and antiparallel ($-$) configurations,
leading to a determination of the neutron electric dipole moment
$d_n$.  Here $\gamma_n$ is the neutron gyromagnetic ratio and $h$ is
Planck's constant.

A crucial aspect of the experiment is that the magnetic field be
continuously monitored so that any drifts can be corrected.  To
monitor the field, a second ``comagnetometer'' atomic species is
stored in the cell and its spin precession frequency is measured
optically.  Normally $^{199}$Hg is used for this
purpose~\cite{bib:green,bib:hg-better} in part because its own EDM has
been constrained to be small~\cite{bib:hgupdate}.

Ideally, the magnetic field $B_0$ should be uniform so that long
free-precession times can be achieved for both the neutrons and Hg
atoms.  In the previous most precise nEDM
experiment~\cite{bib:baker,bib:pendlebury} it was found that a
vertical gradient $\partial B_{0z}/\partial z$ in the magnetic field
induced false EDM's for the neutrons and Hg atoms.  The frequency
shift can be thought of as a Bloch-Siegert shift for particles
traveling in bound orbits within the trap~\cite{bib:gpe1,bib:gpe2}.
The ultracold neutrons traverse the measurement volume slowly enough
that their accrued phase can be thought of as a geometric
phase~\cite{bib:gpe1}.

The false EDM's can also be calculated by correlation function
techniques~\cite{bib:gpe3,bib:gpe4}.  This led to the realization that
false EDM for the Hg atoms could written in the form
\begin{equation}
\label{eq:pignol}
  d_{f,\rm Hg}=-\frac{\hbar\gamma^{2}}{2c^{2}}\langle xB_x+yB_y\rangle
\end{equation}
using integration by parts~\cite{bib:pignol-roccia}.  Here, $\gamma$ is the gyromagnetic ratio of Hg, and the average is over the containing volume. This
form is valid to high precision, even reproducing higher-order effects
first studied using Monte Carlo techniques in Ref.~\cite{bib:gpe2}.

The work presented here concerns higher-order corrections beyond the
first-order vertical gradient $\partial B_{0z}/\partial z$.  We show
that for spherical traps, all higher-order terms contributing to
Eq.~(\cite{eq:pignol}) are identically zero.  We further show that for
the usual cylindrical traps used in EDM experiments, selection of
quasi-spherical dimensions for the trap can reduce the higher-order
contributions to false EDM of the Hg atoms.


\section{Harmonic Decomposition of the Magnetic Field}

Within the measurement region of EDM experiments, nonmagnetic
components are used so that the field can be measured and controlled
precisely.  Since bound and free currents are absent from this region,
the magnetic field can be written as~\cite{bib:jackson}
\begin{equation}
\vec{B}=-\nabla\Phi_M
\end{equation}
where $\Phi_M$ is the magnetic scalar potential.  Since
$\nabla\cdot\vec{B}=0$, $\Phi_M$ obeys Laplace's equation:
\begin{equation}
\nabla^2\Phi_M=0.
\end{equation}
The general solution for a boundary-value problem in can be written in
terms of spherical harmonics $Y_{\ell m}(\theta,\phi)$ as
\begin{equation}\label{eq:boundaryvalue}
  \Phi_M(r,\theta,\phi)=\sum_{\ell=0}^\infty\sum_{m=-\ell}^\ell\left[A_{\ell m}r^\ell+B_{\ell m}r^{-(\ell+1)}\right]Y_{\ell m}(\theta,\phi)
\end{equation}
with the spherical harmonics
\begin{equation}
    Y_{\ell,m}(\theta,\phi)=\sqrt{\frac{2\ell+1}{4\pi}\frac{(\ell-m)!}{(\ell+m)!}}P_{\ell}^{m}(\cos\theta)e^{im\phi},
\end{equation}
and the associated Legendre polynomials, $P_{\ell}^{m}$. define things and state that $\Phi$ must be real.  If we
define $r=0$ to be the center of the trap, the requirement that
$\vec{B}$ remain finite enforces $B_{\ell m}=0$.

The average appearing in Eq.~(\ref{eq:pignol}) can be recast in terms
of the scalar potential as
\begin{equation}
  \langle xB_x+yB_y\rangle=-\left\langle\left(x\frac{\partial}{\partial
    x}+y\frac{\partial}{\partial y}\right)\Phi_M\right\rangle
\end{equation}
which in terms of the spherical harmonics becomes
\begin{equation}
  \label{eq:xb}
  \langle xB_x+yB_y\rangle=-\sum_{\ell=0}^\infty\sum_{m=-\ell}^\ell
  A_{\ell m}\left\langle\left(x\frac{\partial}{\partial
    x}+y\frac{\partial}{\partial y}\right)r^\ell Y_{\ell
    m}\right\rangle
\end{equation}
where again the average is conducted over the measurement cell.

\section{Application of the Wigner-Eckart Theorem}

We can apply the Wigner-Eckart Theorem to this system by making an
analogy to matrix elements in quantum mechanics.  The differential
operator in Eq.~(\ref{eq:xb}) $\left(x\frac{\partial}{\partial
  x}+y\frac{\partial}{\partial y}\right)$ can be analogized to the
operator $(xp_x+yp_y)$ in quantum mechanics.

For a particle in a spherically symmetric potential, the stationary
states may be written as eigenstates of the angular momentum operators
\begin{equation}
L^2|n\ell m\rangle=\hbar^2\ell(\ell+1)|n\ell m\rangle
\end{equation}
\begin{equation}
L_z^2|n\ell m\rangle=\hbar m|n\ell m\rangle
\end{equation}
where $|n\ell m\rangle$ represents the stationary state and $\ell$ and
$m$ are quantum numbers.  The quantum number $n$ would count energy
levels.  In the position representation, the states factorize in
spherical coordinates as
\begin{equation}
\langle\vec{r}|n\ell m\rangle=R_{n\ell}(r)Y_{\ell m}(\theta,\phi)
\end{equation}
and thus the states are related to the spherical harmonics.

Eq.~(\ref{eq:xb}) therefore has a number of elements which bear a
strong similarity with calculations in quantum mechanics.  The average
in \ref{eq:xb} is over the cell...it has to be a spherical cell.  The
term under the average is proportional to a quantum mechanical matrix
element as
\begin{equation}
\label{eq:matrix}
  \left\langle\left(x\frac{\partial}{\partial
  x}+y\frac{\partial}{\partial y}\right)r^\ell Y_{\ell
  m}\right\rangle\sim\langle n'00|(xp_x+yp_y)|n\ell m\rangle
\end{equation}
where we have inserted a spherically symmetric state with
$\ell'=m'=0$.  Being constructed from products of vector operators, we
will show that the operator $(xp_x+yp_y)$ can be written as an
admixture of spherical tensors of rank 0 and 2.

The Wigner-Eckart theorem relates matrix elements of spherical tensors
involving these states to Clebsch-Gordan coefficients
\cite{bib:sakurai}.  Applied to two states $|n'\ell'm'\rangle$ and
$|n\ell m\rangle$, it would read
\begin{equation}
\langle n'\ell'm'|T_q^{(k)}|n\ell m\rangle=\langle\ell k;mq|\ell k;\ell'm'\rangle\frac{\langle n'\ell'||T^{(k)}||n\ell\rangle}{\sqrt{2j+1}}
\end{equation}
where $T_q^{(k)}$ is a spherical tensor of rank $k$ with magnetic
quantum number $q$.  The double-bar matrix element is a reduced matrix
element that does not depend on $m$, $m'$, or $q$.

The Clebsch-Gordan coefficient $\langle\ell k;mq|\ell
k;\ell'm'\rangle$ can be thought of in the following more familiar
way.  Imagine two angular momentum operators $\vec{L}$ and $\vec{K}$
with a simultaneous eigenstate $|\ell k;mq\rangle$ s.t.
\begin{equation}
L^2|\ell k;mq\rangle=\hbar^2\ell(\ell+1)|\ell k;mq\rangle
\end{equation}
\begin{equation}
L_z|\ell k;mq\rangle=\hbar m|\ell k;mq\rangle
\end{equation}
\begin{equation}
K^2|\ell k;mq\rangle=\hbar^2k(k+1)|\ell k;mq\rangle
\end{equation}
\begin{equation}
K_z|\ell k;mq\rangle=\hbar q|\ell k;mq\rangle.
\end{equation}
If we now define a new operator $\vec{L}'=\vec{L}+\vec{K}$, the
Clebsch-Gordan coefficients represent the coefficients transforming to
the new basis $|\ell k;\ell'm'\rangle$ where
\begin{equation}
L^2|\ell k;mq\rangle=\hbar^2\ell(\ell+1)|\ell k;\ell'm'\rangle
\end{equation}
\begin{equation}
K^2|\ell k;mq\rangle=\hbar^2k(k+1)|\ell k;\ell'm'\rangle
\end{equation}
\begin{equation}
L'^2|\ell k;mq\rangle=\hbar^2\ell'(\ell'+1)|\ell k;\ell'm'\rangle
\end{equation}
\begin{equation}
L'_z|\ell k;mq\rangle=\hbar m'|\ell k;\ell'm'\rangle.
\end{equation}
In this way, the Clebsch-Gordan coefficient is related to the addition
of angular momentum of the state $|n\ell m\rangle$ to that of the
spherical tensor $T_q^{(k)}$ and reaching the state
$|n'\ell'm'\rangle$.

The Clebsch-Gordan coefficient is only non-zero if
\begin{equation}
  \label{eq:z}
  m+q=m'
\end{equation}
and
\begin{equation}
\label{eq:ell}
  |\ell-k|\leq\ell'\leq\ell+k
\end{equation}

As mentioned earlier the operator $(xp_x+yp_y)$ can be written as an
admixture of spherical tensors of rank 0 and 2.  The following two
spherical tensors may be constructed from the Cartesian components of
the vector operators $\vec{r}$ and $\vec{p}$:
\begin{equation}
T_0^{(0)}=-\frac{\vec{r}\cdot\vec{p}}{3}=\frac{-xp_x-yp_y-zp_z}{3}
\end{equation}
and
\begin{equation}
T_0^{(2)}=\frac{3zp_z-(\vec{r}\cdot\vec{p})^2}{\sqrt{6}}=\frac{-xp_x-yp_y+2zp_z}{\sqrt{6}}.
\end{equation}
The operator of interest can then be written as
\begin{equation}
xp_x+yp_y=-2T_0^{(0)}-\sqrt{\frac{2}{3}}T_0^{(2)}
\end{equation}
which are spherical tensors of rank 0 and 2 with magnetic quantum
number $q=0$.

Finally we note that because these operators are even under the parity
transformation, they may only link states of the same parity.

We can now apply Eqs.~(\ref{eq:z}) and (\ref{eq:ell}) to
Eq.~(\ref{eq:xb}) to find that only terms in the sums with $m=0$ and
$\ell=0,1,2$ will contribute.  Since the operator in Eq.~(\ref{eq:xb})
is a differential operator, the term with $\ell=0$ also cannot
contribute.  The term with $\ell=1$ is ruled out by the parity
selection rule.  We are then left with only one non-zero term arising
from the harmonic decomposition of the field:
\begin{equation}
\langle
xB_x+yB_y\rangle=-A_{20}\left\langle\left(x\frac{\partial}{\partial
  x}+y\frac{\partial}{\partial y}\right)r^2 Y_{20}\right\rangle
\end{equation}
where the average can be readily carried out since $r^2Y_{20}$ is a
polynomial of degree 2 in the Cartesian coordinates.

\section{Cylindrical Trap and Suppression of Higher Orders}
\label{sec:cylinder}

A requirement of this calculation is that the EDM cell be spherical,
which is not an attractive option for EDM experiments.  A more typical
geometry is a cylindrically symmetric geometry.  In this case, it can
still readily be demonstrated that only $m=0$ terms contribute in the
second sum in Eq.~(\ref{eq:xb}).  Switching to cylindrical
$(\rho,\phi,z)$ coordinates, the differential operator becomes
$x\frac{\partial}{\partial x}+y\frac{\partial}{\partial
  y}=\rho\frac{\partial}{\partial\rho}$ and since $Y_{\ell m}\sim
e^{im\phi}$ these terms would average to zero unless $m=0$.
Furthermore, terms with $\ell={\rm odd}$ are odd in $z$.  Thus if the
cylindrical cell is centered on $z=0$, these terms will also average
to zero.  We are then left with
\begin{equation}
  \label{eq:cyl}
  \langle xB_x+yB_y\rangle=-\sum_{\ell=2,4,6,...}^\infty A_{\ell
    0}\left\langle\left(\rho\frac{\partial}{\partial
    \rho}\right)r^\ell Y_{\ell
    0}\right\rangle
\end{equation}
In order to more easily compare to~\cite{bib:pignol-roccia}, we introduce a different and convenient normalization of the polynomials, and define
%Need a better name than Sigma, I think.
\begin{equation}
    \Sigma_{\ell,m} = \frac{1}{\ell}r^{\ell}P_{\ell}^{m}(\cos\theta).
\end{equation}
Equation \ref{eq:cyl} can then be written:
\begin{equation}
    \langle xB_x+yB_y\rangle= - \sum_{\ell=2,4,6...}^{\infty} g_{\ell0}\left\langle\left(\rho\frac{\partial}{\partial \rho}\right)\Sigma_{\ell0}\right\rangle,
\end{equation}
where $g_{\ell0}$ takes the place of $A_{\ell0}$ in equation \ref{eq:boundaryvalue}.  This normalization ensures that the $\ell=0$ components of $B_{z}$ are polynomials containing $z^{\ell+1}$.
The purpose of this section is to demonstrate that $\ell=2$ dominates,
that for a certain choice of cell dimensions the $\ell=4$ term can be
zeroed, and that for this selection the $\ell=6$ term can be reduced
compared to the typical cell geometry used for the ILL nEDM
experiment.
The Legendre polynomials of interest are
\begin{equation}
P_{2}^{0}(\cos\theta)=\frac{1}{2}(3\cos^{2}\theta-1)
\end{equation}
\begin{equation}
P_{4}^{0}(\cos\theta)=\frac{1}{8}(35\cos^{4}\theta-30\cos^{2}\theta+3)
\end{equation}
and
\begin{equation}
P_{6}^{0}(\cos\theta)=\frac{1}{16}(231\cos^{6}\theta-315\cos^{4}\theta+105\cos^{2}\theta-5).
\end{equation}
In general $\Sigma_{\ell 0}$ is a polynomial in $\rho$ and $z$ when
expressed in cylindrical coordinates.  For the terms of interest, the
polynomials are
\begin{equation}
\Sigma_{20}=\frac{1}{4}\left(2z^2-\rho^2\right)
\end{equation}
\begin{equation}
\Sigma_{40}=\frac{1}{32}\left(8z^{4}-24z^{2}\rho^{2}+3\rho^{4}\right)
\end{equation}
and
\begin{equation}
\Sigma_{60}=\frac{1}{96}\left(16z^{6}-120z^{4}\rho^{2}+90z^{2}\rho^{4}-5\rho^{6}\right) .
\end{equation}
We define the origin of coordinates to lie at the center of a
measurement cell with height $H$ and radius $R$.  Carrying out the
average over the cell, the first three non-zero terms of
Eq.~(\ref{eq:cyl}) become
\begin{multline}
\label{eq:xbc}
  \langle
  xB_x+yB_y\rangle=g_{20}\left(\frac{R^{2}}{4}\right)+g_{40}\left(\frac{R^{2}H^{2}}{16}-\frac{R^{4}}{8}\right)\\
  +g_{60}\left(\frac{5R^{2}H^{4}}{64}-\frac{5R^{4}H^{2}}{16}+\frac{5R^{6}}{64}\right)...
\end{multline}
The expression for $d_{f,Hg}$ then becomes\\

\begin{multline}
    d_{f,Hg} = -\frac{\hbar\gamma_{Hg}^{2}}{2c^{2}}\left[g_{20}\left(\frac{R^{2}}{4}\right)+g_{40}\left(\frac{R^{2}H^{2}}{16}-\frac{R^{4}}{8}\right)\right.\\
    +\left.g_{60}\left(\frac{5R^{2}H^{4}}{64}-\frac{5R^{4}H^{2}}{16}+\frac{5R^{6}}{64}\right)+...\right]
\end{multline}
We make the following observations:

... 20 term agrees with Pignol and Roccia

The $g_{20}$ term is in agreement with Ref.~\cite{bib:pignol}.  In
Ref.~\cite{bib:pignol}, the scalar potential was expanded to second
order and it was noted for example that
\begin{equation}
  \langle xB_x+yB_y\rangle=-\frac{R^2}{4}\left\langle\frac{\partial B_z}{dz}\right\rangle.
\end{equation}
Using our expression for the $A_{20}$ term alone indeed expresses a
uniform gradient
\begin{equation}
  \left(\frac{\partial B_z}{\partial
    z}\right)_{20}=-4\sqrt{\frac{5}{16\pi}}A_{20}
\end{equation}
and therefore our expression in Eq.~(\ref{eq:xbc}) is in agreement with
Ref.~\cite{bib:pignol} to this order.

For ILL R is 0.235 m, H is 0.12 m, and so the $\ell=4$ term is $\num{3.3e-4}$, and the $\ell=6$ term is $\num{-8.4e-6}$.  However, the $g_{40}$ term in equation \ref{eq:xbc} can be made to be zero by careful selection of R and H.  Keeping approximately the same volume, we can set R = 17, H = $\sqrt{2}\cdot$17 = 24.  The $\ell=4$ term is zero, and the $\ell=6$ term is \num{2.0e-6}, a factor of 4 suppression.


\section{Conclusion}

...


\begin{figure}[h]
\centering\includegraphics[width=0.4\linewidth]{placeholder}
\caption{Figure caption}
\end{figure}


%% The Appendices part is started with the command \appendix;
%% appendix sections are then done as normal sections
%% \appendix

%% \section{}
%% \label{}

%% References
%%
%% Following citation commands can be used in the body text:
%% Usage of \cite is as follows:
%%   \cite{key}          ==>>  [#]
%%   \cite[chap. 2]{key} ==>>  [#, chap. 2]
%%   \citet{key}         ==>>  Author [#]

%% References with bibTeX database:

\begin{table}[h]
\centering
\begin{tabular}{l l l}
\hline
\textbf{Treatments} & \textbf{Response 1} & \textbf{Response 2}\\
\hline
Treatment 1 & 0.0003262 & 0.562 \\
Treatment 2 & 0.0015681 & 0.910 \\
Treatment 3 & 0.0009271 & 0.296 \\
\hline
\end{tabular}
\caption{Table caption}
\end{table}


\bibliographystyle{model1-num-names}
\bibliography{sample.bib}

%% Authors are advised to submit their bibtex database files. They are
%% requested to list a bibtex style file in the manuscript if they do
%% not want to use model1-num-names.bst.

%% References without bibTeX database:

% \begin{thebibliography}{00}

%% \bibitem must have the following form:
%%   \bibitem{key}...
%%

% \bibitem{}

% \end{thebibliography}


\end{document}

%%
%% End of file `elsarticle-template-1-num.tex'.
